% This is "sig-alternate.tex" V2.1 April 2013
% This file should be compiled with V2.5 of "sig-alternate.cls" May 2012
%
% This example file demonstrates the use of the 'sig-alternate.cls'
% V2.5 LaTeX2e document class file. It is for those submitting
% articles to ACM Conference Proceedings WHO DO NOT WISH TO
% STRICTLY ADHERE TO THE SIGS (PUBS-BOARD-ENDORSED) STYLE.
% The 'sig-alternate.cls' file will produce a similar-looking,
% albeit, 'tighter' paper resulting in, invariably, fewer pages.
%
% ----------------------------------------------------------------------------------------------------------------
% This .tex file (and associated .cls V2.5) produces:
%       1) The Permission Statement
%       2) The Conference (location) Info information
%       3) The Copyright Line with ACM data
%       4) NO page numbers
%
% as against the acm_proc_article-sp.cls file which
% DOES NOT produce 1) thru' 3) above.
%
% Using 'sig-alternate.cls' you have control, however, from within
% the source .tex file, over both the CopyrightYear
% (defaulted to 200X) and the ACM Copyright Data
% (defaulted to X-XXXXX-XX-X/XX/XX).
% e.g.
% \CopyrightYear{2007} will cause 2007 to appear in the copyright line.
% \crdata{0-12345-67-8/90/12} will cause 0-12345-67-8/90/12 to appear in the copyright line.
%
% ---------------------------------------------------------------------------------------------------------------
% This .tex source is an example which *does* use
% the .bib file (from which the .bbl file % is produced).
% REMEMBER HOWEVER: After having produced the .bbl file,
% and prior to final submission, you *NEED* to 'insert'
% your .bbl file into your source .tex file so as to provide
% ONE 'self-contained' source file.
%
% ================= IF YOU HAVE QUESTIONS =======================
% Questions regarding the SIGS styles, SIGS policies and
% procedures, Conferences etc. should be sent to
% Adrienne Griscti (griscti@acm.org)
%
% Technical questions _only_ to
% Gerald Murray (murray@hq.acm.org)
% ===============================================================
%
% For tracking purposes - this is V2.0 - May 2012

\documentclass{sig-alternate-05-2015}


\begin{document}

% Copyright
\setcopyright{acmcopyright}
%\setcopyright{acmlicensed}
%\setcopyright{rightsretained}
%\setcopyright{usgov}
%\setcopyright{usgovmixed}
%\setcopyright{cagov}
%\setcopyright{cagovmixed}


% DOI
\doi{10.475/123_4}

% ISBN
\isbn{123-4567-24-567/08/06}

%Conference
\conferenceinfo{PLDI '13}{June 16--19, 2013, Seattle, WA, USA}

\acmPrice{\$15.00}

%
% --- Author Metadata here ---
\conferenceinfo{WOODSTOCK}{'97 El Paso, Texas USA}
%\CopyrightYear{2007} % Allows default copyright year (20XX) to be over-ridden - IF NEED BE.
%\crdata{0-12345-67-8/90/01}  % Allows default copyright data (0-89791-88-6/97/05) to be over-ridden - IF NEED BE.
% --- End of Author Metadata ---

\title{Runtime comparison solving Gray-Scott equation on different OpenCL devices}
%\subtitle{[Extended Abstract]
%\titlenote{A full version of this paper is available as
%\textit{Author's Guide to Preparing ACM SIG Proceedings Using
%\LaTeX$2_\epsilon$\ and BibTeX} at
%\texttt{www.acm.org/eaddress.htm}}}
%
% You need the command \numberofauthors to handle the 'placement
% and alignment' of the authors beneath the title.
%
% For aesthetic reasons, we recommend 'three authors at a time'
% i.e. three 'name/affiliation blocks' be placed beneath the title.
%
% NOTE: You are NOT restricted in how many 'rows' of
% "name/affiliations" may appear. We just ask that you restrict
% the number of 'columns' to three.
%
% Because of the available 'opening page real-estate'
% we ask you to refrain from putting more than six authors
% (two rows with three columns) beneath the article title.
% More than six makes the first-page appear very cluttered indeed.
%
% Use the \alignauthor commands to handle the names
% and affiliations for an 'aesthetic maximum' of six authors.
% Add names, affiliations, addresses for
% the seventh etc. author(s) as the argument for the
% \additionalauthors command.
% These 'additional authors' will be output/set for you
% without further effort on your part as the last section in
% the body of your article BEFORE References or any Appendices.

\numberofauthors{1} %  in this sample file, there are a *total*
% of EIGHT authors. SIX appear on the 'first-page' (for formatting
% reasons) and the remaining two appear in the \additionalauthors section.
%
\author{
% You can go ahead and credit any number of authors here,
% e.g. one 'row of three' or two rows (consisting of one row of three
% and a second row of one, two or three).
%
% The command \alignauthor (no curly braces needed) should
% precede each author name, affiliation/snail-mail address and
% e-mail address. Additionally, tag each line of
% affiliation/address with \affaddr, and tag the
% e-mail address with \email.
%
% 1st. author
\alignauthor
Michael Quell\\
       \affaddr{Institute of Analysis and Scientific Computing}\\
       \affaddr{Wiedner Hauptstra\ss e 8-10}\\
       \affaddr{Vienna, Austria}\\
       \email{michael.quell@yahoo.de}
}
\maketitle
\begin{abstract}
n example of a reaction-diffusion equation with chaotic solutions. You can expect patterns to emerge from chaos. A uniformly discretization in space and periodic boundary conditions allows the Fast Fourier Transform to be used, so that when coupled with a suitable time stepping scheme a numerical method that suits the parallelism of OpenCL is obtained. The code was benchmarked on various CPU and GPU devices. Performance results for various problem sizes are shown. Example code can be found at:\\
\url{https://github.com/MichaelQuell/GrayScott-OpenCl}
\end{abstract}


%
% The code below should be generated by the tool at
% http://dl.acm.org/ccs.cfm
% Please copy and paste the code instead of the example below. 
%
 \begin{CCSXML}
<ccs2012>
<concept>
<concept_id>10010147.10010341.10010349.10010362</concept_id>
<concept_desc>Computing methodologies~Massively parallel and high-performance simulations</concept_desc>
<concept_significance>300</concept_significance>
</concept>
<concept>
<concept_id>10010405.10010432.10010436</concept_id>
<concept_desc>Applied computing~Chemistry</concept_desc>
<concept_significance>300</concept_significance>
</concept>
</ccs2012>
\end{CCSXML}

\ccsdesc[300]{Computing methodologies~Massively parallel and high-performance simulations}
\ccsdesc[300]{Applied computing~Chemistry}


%
% End generated code
%

%
%  Use this command to print the description
%
\printccsdesc

% We no longer use \terms command
%\terms{Theory}

\keywords{Gray-Scott, Splitting, OpenCl, FFT }

\section{Introduction}
The Gray-Scott-equation\cite{grayscott} describes the reaction of 2 chemicals and it is given by 
\vspace{1mm}
\begin{align*}
\frac{\partial u}{\partial t}&= D_u\Delta u-uv^2+F(1-u),\\
\frac{\partial v}{\partial t}&= D_v\Delta v+uv^2-(F+k)v.
\end{align*}
\vspace{-1.75mm}$~$\\
The equation is solved on $[-4\pi,4\pi]^2$ with periodic boundary conditions.
The parameters used in the simulations were $D_u=0.04$, $D_v=0.005$, $F=0.038$ and $k=0.076$. The initial data is a 2-dimensional Gaussian function in both components.

\section{The {\secit Body} of The Paper}
The equation is solved using a splitting method, in this case we split the equation into the non linear 
\begin{eqnarray*}
\frac{\partial u}{\partial t}&=& -uv^2 ,\quad \frac{\partial v}{\partial t}=uv^2 \\
\end{eqnarray*}
and the linear part
\begin{eqnarray*}
\frac{\partial u}{\partial t}&=& D_u\Delta u+F(1-u), \\
\frac{\partial v}{\partial t}&=& D_v\Delta v-(F+k)v. \\
 \end{eqnarray*} 
 %\vspace{-5}
 The linear part can be solved exactly in Fourierspace and the non linear could be exactly solved using the Lambert $W$ function, but there is no implementation available for OpenCl, instead fix point iteration is used to solve the implicit midpoint rule. The two parts are combined with strang splitting. The space discretization is equidistant in both axes.

\subsection{Implementation}
\begin{enumerate}
\item Initialize the data
\item time stepping
\begin{enumerate}
\item Call non linear kernel
\item {\bf Compute FFT }\cite{clFFT}
\item Call linear kernel
\item {\bf Compute iFFT }
\item Call non linear kernel
\item Do some output
\end{enumerate}
\item Shut down the program
\end{enumerate}
The linear and non linear kernel operates on each grid point individually and are perfectly suit the parallelism of OpenCl. The most time consuming step is the computation of the FFT and iFFT. Also there is only data transfers from the GPU, when you do some output.



\section{Conclusions}
The results show that the GPU's outperform the CPU's in single precision, but when computing double precision the performance of the CPU's stays the same, while the GPU's have a significant drop. To compute the equation on larger grids or in 3-dimension, you will, have to face the problem that on GPU's memory is short. One could avoid that by writing a distributed FFT as the other kernels are independent on the problem size. 

%\end{document}  % This is where a 'short' article might terminate

%ACKNOWLEDGMENTS are optional
\section{Acknowledgments}
I would like to thank Benson  {{K}}. Muite  for funding the hardware for the research.

%
% The following two commands are all you need in the
% initial runs of your .tex file to
% produce the bibliography for the citations in your paper.
\bibliographystyle{abbrv}
\begin{thebibliography}{1}\itemsep=-0.01em
  \setlength{\baselineskip}{1.0em}
  \bibitem{grayscott}
P.~Gray, S.~Scott, Chemical Waves and Instabilities, Clarendon, Oxford, 1990.

\bibitem{clFFT}
  clFFT open source library to compute FFT with OpenCl\\
  \url{https://github.com/clMathLibraries/clFFT}
  
\bibitem{splitting} maybe cite something for splitting ???
\end{thebibliography}
%\bibliography{sigproc}  % sigproc.bib is the name of the Bibliography in this case
% You must have a proper ".bib" file
%  and remember to run:
% latex bibtex latex latex
% to resolve all references
%
% ACM needs 'a single self-contained file'!
%
%APPENDICES are optional
%\balancecolumns

%\balancecolumns % GM June 2007
% That's all folks!
\end{document}
